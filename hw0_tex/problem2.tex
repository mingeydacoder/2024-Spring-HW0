\documentclass{article}
\setcounter{section}{-1}

% Language setting
% Replace `english' with e.g. `spanish' to change the document language
\usepackage[english]{babel}
\usepackage{CJKutf8}
% Set page size and margins
% Replace `letterpaper' with `a4paper' for UK/EU standard size
\usepackage[letterpaper,top=2cm,bottom=2cm,left=3cm,right=3cm,marginparwidth=1.75cm]{geometry}

% Useful packages
\usepackage{amsmath}
\usepackage{graphicx}
\usepackage[colorlinks=true, allcolors=blue]{hyperref}
\usepackage{setspace}

\title{Introduction to FinTech HW0 Problem2}
\author{B10209040\quad陳彥倫}

\begin{document}
\begin{CJK*}{UTF8}{bsmi}
\maketitle

\begin{spacing}{2}
    \subsection*{1. Blockchain Basics}
    gas 可以視為在區塊鏈上執行任何操作時所支付的手續費。其價格通常使用Gwei=$10^{-9}$ETH作為單位,依據調整數
    量的多寡決定交易的速度。因此每個用戶都會設置Gas limit,以避免無上限的燃料費導致消耗大量資源而出現錯誤的情況
    。由上述可知一筆交易之手續費即為 Gas Limit * Gas Price 的結果。\\
    EIP-1559 為一改變以太坊交易手續費機制的提案,其目的是為了解決以下問題:a. gas的波動性,可能會使用戶無法交易
    b. 在EIP-1599出現之前,gas價格由一拍賣機制決定,網路繁忙時gas費用將會急劇上升c.gas價格之高將迫使開發者轉
    投其他區塊鏈。因此EIP-1559將gas區分為兩個部分,基礎費及小費,以根據使用量進行動態調整來解決以太坊交易擁堵的問題。
\end{spacing}

\begin{spacing}{2}
    \subsection*{2. Smart Contract Basics} 
    var1 - X, var2 - slot 0, var3 - X, var4 - slot 1, var5 - slot 1, var6 - slot 1, var7 - slot 2
    var8 - 3
\end{spacing}

\begin{spacing}{2}
    \subsection*{3. Defi Basics} 
    無償損失意為在進行流動性挖礦時,因幣價的變化所帶來的損失,且幣價變化越大,損失也越大。假如A幣與B幣兩中資產需能
    在去中心化交易所中自由兌換,則需要流動性池的存在。在兩資產價值相同的情況下,某人想以A兌換B,即代表將A放入池中
    並拿取B。但經濟學告訴我們,需求上升會帶動價格上升;供應上升會帶動價格下跌。當此人拿出 A 去換 B 的那一
    刻,A 的供應增加而 B 的需求增加,因此 A 的價格會跌,B 的價格會漲,以程式碼和 LP 自動實現以上供求與價格的關係,
    就是自動造市商AMM之機制。
    
\end{spacing}


\end{CJK*}
\end{document}